\documentclass{article}
\usepackage{amsmath}
\usepackage{amssymb}
\usepackage{amsthm}

\newtheorem*{theorem}{Theorem}
\newtheorem*{lemma}{Lemma}
\begin{document}

\begin{lemma} If $(\alpha, \alpha)(\beta, \beta) \neq 0$ and $|\alpha| = |\beta|$, then $\alpha = \beta$. \end{lemma}
\begin{proof}
Without loss of generality, suppose $\alpha$ is a prefix of $\beta$. Then $\alpha = \beta \beta'$ for some $\beta$. But
$|\alpha| = |\beta|$, so $|\beta'| = 0$, meaning $\beta' = s(\beta)$. Thus $\alpha = \beta s(\beta) = \beta$.
\end{proof}
An immediate consequence of this is that for any given length, a filter contains at most one element corresponding
to a path of that length. Furthermore, if $\zeta$ is a filter, and $(\alpha, \alpha) \in \zeta$ such that $|\alpha| = n$,
we can restrict $\alpha$ to any length $0 \leq m < n$ to produce a unique element of length $m$ inside $\zeta$.

\begin{theorem} $\hat{E_0} = \{\xi_\alpha$: $\alpha \in E^*\} \cup \{\eta_x$: $x \in E^\infty\}$ \end{theorem}
\begin{proof}
First we show that $\xi_\alpha$ and $\eta_x$ are filters. Let $\alpha$ be a finite path and let $x$ be an infinite path.
Consider $(\beta, \beta), \ (\gamma, \gamma) \in \xi_\alpha$ with $|\beta| = m$, $|\gamma| = n$. 
Reading edges left to right, $\beta$ contains the first $m$ edges of $\alpha$ and $\gamma$ 
contains the first $n$ edges of $\alpha$. Without loss of generality, assume $m \leq n$. 
We can extend $\beta$ by the next $n - m$ edges of $\alpha$ to produce $\gamma$. Thus
$\beta$ is a prefix of $\gamma$, so $(\beta, \beta)(\gamma, \gamma) = (\gamma, \gamma) \in \xi_\alpha$.
A similar argument shows that $\eta_x$ is closed under multiplication. So both $\xi_\alpha$ and $\eta_x$ are prefilters
on $E(S(E))$. Next, let $(\beta, \beta) \in \xi_\alpha$, and let $(\delta, \delta) \in E(S(E))$ so that 
$(\beta, \beta) \leq (\delta, \delta)$. Then $\beta = \delta \delta'$ for some $\delta' \in E^*$. 
Thus $\alpha = \beta \beta' = \delta \delta' \beta'$. Therefore $\delta$ is a prefix of $\alpha$, so 
$(\delta, \delta) \in \xi_\alpha$. Like before, a very similar argument works for $\eta_x$. So both 
$\xi_\alpha$ and $\eta_x$ are filters. Going the other way, take a filter $\zeta \in \hat{E_0}$.
We do this in cases.
\\ \\
    \textit{Case $1$. $\zeta$ is finite.}
Let $(\alpha, \alpha)$ be the minimum element inside $\zeta$. Since $\zeta$ is a filter,
For $(\beta, \beta) \in E(S(E))$, $(\alpha, \alpha) \leq (\beta, \beta) \implies (\beta, \beta) \in \zeta$.
However, by the definition of the minimal element, $(\beta, \beta) \in \zeta \implies (\alpha, \alpha) \leq (\beta, \beta)$.
Thus $(\beta, \beta) \in \zeta$ if and only if $(\alpha, \alpha) \leq (\beta, \beta) \iff \beta$ is a prefix of $\alpha$. So $\zeta = \xi_\alpha$.
\\ \\
    \textit{Case $2$. $\zeta$ is infinite.}
By the lemma above, we can get an idea of what elements of $\zeta$ look like.
For each nonnegative integer, $\zeta$ contains precisely one element corresponding
to a path of that length. Because we require nonzero product between elements, every
path in the filter is a prefix of every longer path also contained inside the filter.
All of this given, we can find an $x \in E^\infty$ such that every path inside $\zeta$
is a prefix of $x$. By the uniqueness of filter elements, it follows that $\zeta = \xi_\alpha$.
\\ \\
We have shown that a nonempty subset $\zeta \subset E(S(E))$ is a filter if and only if
it is of the form $\xi_\alpha$ or $\eta_x$. Thus $\hat{E_0} = \{\xi_\alpha$: $\alpha \in E^*\} \cup \{\eta_x$: $x \in E^\infty\}$.
\end{proof}

\begin{theorem} Let $E$ be a directed graph, and $\alpha \in E^*$ such that $|r^{-1}\{s(\alpha)\}| = \infty$.
Let $X, Y \subseteq_{\text{fin}} E(S(E))$, and $Z$ be a finite cover of $E^{X, Y}$.
If $\xi_\alpha \in \mathcal{U}(X, Y)$, then $\xi_\alpha \cap Z \neq \emptyset$. \end{theorem}
\begin{proof}
First note:
\begin{align*}
    E^{X, Y} &= \{ e \in E(S(E))\text{: } e \leq x \ \forall x \in x \text{ and } ey = 0 \ \forall y \in Y \} \\
             &= \{ e \in E(S(E))\text{: } e \leq \ \text{min}(X) \text{ and } ey = 0 \ \forall y \in Y \} \\
             &= E^{\{\text{min}(X)\}, Y}
\end{align*}
Letting min$(X) = (x, x)$,
\begin{align*}
    E^{X, Y} = \{ (xx', xx') \text{: } x' \in E^*, \ r(x') = s(x) \text{ and } (xx', xx')y = 0 \ \forall y \in Y \}
\end{align*}

Consider the set $C := \{ (\alpha b, \alpha b)\text{: } b \in E^1, \ s(\alpha) = r(b) \}$. By the assumption
that $s(\alpha)$ is an infinite receiver, $C$ is infinite. Given $y \in Y$, let $\nu$ be the path corresponding to $y$.
Since $\xi_\alpha \in \mathcal{U}(X, Y)$, $\nu$ is not a prefix of $\alpha$, and thus not a proper prefix of $\alpha b$ 
for any $b$. Thus, if $(\alpha b, \alpha b)y \neq 0$, $\alpha b$ is a prefix of $\nu$. Then for $\beta \neq b$, 
$\alpha \beta$ cannot be a prefix of $\nu$. So there is at most one element of $C$ such that ($\alpha b, \alpha b)y \neq 0$.
By the assumption that $Y$ is finite, all but finitely many elements of $C$ are inside $E^{\{(x, x)\}, Y}$. Therefore, if $Z$
is a cover of $E^{X, Y}$, $Z$ is an outer cover of the infinite set $E^{X, Y} \cap C$. Because $Z$ is finite, $\exists z \in Z$ 
with $(\alpha b, \alpha b)z \neq 0$ for infinitely many $(\alpha b, \alpha b) \in E^{X, Y} \cap C$. If $\upsilon$ is the path corresponding to $z$,
then for every $b$, either $\upsilon$ is a prefix of $\alpha b$, or $\alpha b$ is a prefix of $\upsilon$.
All the $\alpha b$ are the same length with a different starting edge, so if one is a prefix of $\upsilon$,
no other can be a prefix of $\upsilon$. So $\upsilon$ is a prefix of $\alpha b$ for infinitely many $b$. Thus $|\upsilon| \leq |\alpha| + 1$.
If $|\upsilon| = |\alpha| + 1$, we have a contradiction: $b = \beta$ for all $(\alpha b, \alpha b), (\alpha \beta, \alpha, \beta) \in C$.
Thus $|\upsilon| \leq |\alpha|$, so $\upsilon$ is a prefix of $\alpha$. Therefore $z = (\upsilon, \upsilon) \in \xi_\alpha$, so 
$\xi_\alpha \cap Z \neq \emptyset$.
\end{proof}

\end{document}
